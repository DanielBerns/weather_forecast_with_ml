\documentclass[10pt,a4paper]{article}
\usepackage[utf8]{inputenc}
\usepackage[T1]{fontenc}
\usepackage{graphicx}
\author{Daniel Walther Berns}
\title{Title}
\begin{document}
	
\section{Abstract}

TL; DR of paper
 
\begin{enumerate}
	\item What are we trying to do and why is it relevant?
	\item Why is this hard?
	\item How do we solve it (i.e. our contribution!)
	\item How do we verify that we solved it:
	\item Experiments
	\item Theory
\end{enumerate}

\section{Introduction}

Longer version of the Abstract, i.e. of the entire paper



\begin{enumerate}
	\item What are we trying to do and why is it relevant?
	\item Why is this hard?
	\item How do we solve it (i.e. our contribution!)
	\item How do we verify that we solved it:
	\begin{enumerate}
		\item Experimental results
		\item Theory
	\end{enumerate}
	\item Extra space? Future work!
\end{enumerate}

\section{Related Work}    
\begin{enumerate}
	\item Academic siblings of our work, i.e. alternative attempts in literature at trying to solve the same problem.
	\item Goal is to “Compare and contrast” - how does their approach differ in either assumptions or method? If their method is applicable to our problem setting I expect a comparison in the experimental section. If not there needs to be a clear statement why a given method is not applicable.
	\item Just describing what another paper is doing is not enough. We need to compare and contrast.
\end{enumerate}

\section{Background}
Academic Ancestors of our work, i.e. all concepts and prior work that are required for understanding our method.
\subsubsection{Problem setting}
This text introduces the problem setting and notation (Formalism) for our method, and
highlights any specific assumptions that are made that are unusual.

\section{Method}
What we do. Why we do it. All described using the general Formalism introduced in
the Problem Setting and building on top of the concepts / foundations introduced in
Background.

\section{Experimental setup}

How do we test that our stuff works? Introduces a specific instantiation of the Problem Setting and implementation details of this Method


\section{Results and Discussion}

\begin{enumerate}
	\item Shows the results of running Method on our problem described in Experimental Setup. 
	\item Compares to baselines mentioned in Related Work. 
	\item Includes statistics and confidence intervals. 
	\item Includes statements on parameters and other potential issues of fairness. 
	\item Includes ablation studies to show that specific parts of the method are relevant. 
	\item Discusses limitations of the method. 
\end{enumerate}

\section{Conclusion}

We did it. This paper rocks and you are lucky to have read it (i.e. brief recap of the entire paper). 
Also, we’ll do all these other amazing things in the future.


\bibliography{article}

	
\end{document}
