\documentclass[10pt,a4paper]{article}
\usepackage[utf8]{inputenc}
\usepackage[T1]{fontenc}
\usepackage{graphicx}
\usepackage{hyperref}
\author{Daniel Walther Berns}
\title{Draft}
\begin{document}
	\maketitle
	\section{Outline}
	    \subsection{Weather forecasting with machine learning}

In \cite{SSIS2011}, weather forecasting for predicting solar power production, distributed generation renewable energy.

In \cite{OBOD2017}, framework
\begin{enumerate}
	\item Data adquisition
	\item Data processing (ETL)
	\item Semantic rule reasoning layer processes RDF data according the rules and produces inferred data.
	\item The learning layer preprocesses the incoming data, extracts relevant features and applies machine learning methods.
	\item The actions layer is responsible for evaluating the results produced in learning layer
\end{enumerate}

In \cite{weyn2021sub}, subseasonal prediction; timeframes; 

\subsection{Goal}

We need 
\begin{enumerate}
	\item Prediction of wind speed and direction, useful for scheduling production of wind generators.
	\item Cloud cover prediction, useful for sizing solar generators.
	\item Prediction of frost, rain, and humidity, useful for agriculture.
\end{enumerate}

Our goal:
 Establish a system of sustained operation over time, able to acquire new data, updating periodically its weights and thus improving its performance.


\subsection{Objectives}

\begin{enumerate}
	\item Continuous data acquisition.
	\item Development, training and comparison of different models.
	\item Inference in different environments (Cloud, notebook, Raspberry Pi).
	\item Periodic update of models with additional data.
\end{enumerate}

\subsection{Continuous data acquisition}

\begin{enumerate}
	\item Data table (Wind, humidity, pressure, temperature, rain, cloudiness, solar radiation) + Images.
	\item Multiple data sources
	\begin{enumerate}
		\item Own weather station
		\item API: 
		\begin{enumerate} 
			\item \url{https://openweathermap.org} (free), 
			\item \url{https://insights.spire.com/weather-api} (paid)
		\end{enumerate}
	    \item Historical data available for starters \url{https://www.smn.gob.ar/descarga-de-datos}
	\end{enumerate}
    \item Process:
    \begin{enumerate}
    	\item Download raw data
    	\item Pre-processing (cleaning, sorting and formatting)
    	\item Review (plotting and descriptive statistics)
    	\item Control dashboard
    	\item Final processing (data as tensorflow dataset)
    \end{enumerate}

\end{enumerate}
	    \subsection{Why is this topic important?}
	    write
	    \subsection{How could I formulate my hypothesis?}
	    write
	    \subsection{What are my results?}
	    write
	    \subsection{What is my major finding?}
	    write

    \section{Context and structure}
        \subsection{Introduction}
	    write

        \subsubsection{Why is this research important?}
	    write

        \subsubsection{What is known about this topic?}
	    write

        \subsubsection{What are my hypothesis?}
	    write

        \subsubsection{What are my objectives?}
	    write

        \subsection{Methods}
        \subsubsection{What is the object of study?}
	    write

        \subsubsection{What are thw research procedures to apply?}
	    write

        \subsection{Results}
        \subsubsection{What are the most significant results of this work?}
	    write

        \subsubsection{What are the supporting results of this work?}
	    write

        \subsection{Discussion and conclusion}
        \subsubsection{What are the major findings of this work?}
	    write

        \subsubsection{What is the significance of these findings?}
	    write
      
\bibliographystyle{ieeetr}
\bibliography{draft}
    
\end{document}
